\begin{filecontents}{references.bib}
@online{odpo,
  title = {Orbital Debris Program Office (ODPO)},
  author = {{NASA}},
  year = {2025},
  url = {https://orbitaldebris.jsc.nasa.gov/},
  note = {Accessed 2025-10-03}
}

@online{das,
  title = {Debris Assessment Software (DAS)},
  author = {{NASA Orbital Debris Program Office}},
  year = {2024},
  url = {https://www.orbitaldebris.jsc.nasa.gov/mitigation/das.html},
  note = {Accessed 2025-10-03}
}

@online{nasaopendata,
  title = {NASA Open Data Portal},
  author = {{NASA}},
  year = {2025},
  url = {https://data.nasa.gov/},
  note = {Accessed 2025-10-03}
}

@online{worldview,
  title = {NASA Worldview},
  author = {{NASA Earthdata}},
  year = {2025},
  url = {https://worldview.earthdata.nasa.gov/},
  note = {Accessed 2025-10-03}
}

@online{copernicus,
  title = {Copernicus Data Space Ecosystem},
  author = {{European Commission / Copernicus}},
  year = {2025},
  url = {https://dataspace.copernicus.eu/},
  note = {Accessed 2025-10-03}
}

@online{earthexplorer,
  title = {USGS EarthExplorer},
  author = {{USGS}},
  year = {2025},
  url = {https://earthexplorer.usgs.gov/},
  note = {Accessed 2025-10-03}
}

@online{nasa_commercial,
  title = {Commercial Space},
  author = {{NASA}},
  year = {2025},
  url = {https://www.nasa.gov/humans-in-space/commercial-space/},
  note = {Accessed 2025-10-03}
}
\end{filecontents}

\documentclass[11pt,a4paper]{article}
\usepackage[utf8]{inputenc}
\usepackage{microtype}
\usepackage{hyperref}
\usepackage{graphicx}
\usepackage{booktabs}
\usepackage{geometry}
\usepackage{enumitem}
\usepackage{caption}
\usepackage{float}
\usepackage{amsmath}
\usepackage{siunitx}
\usepackage{tabularx}
\usepackage{multirow}
\usepackage{csquotes}
\usepackage[backend=biber,style=authoryear,maxbibnames=10]{biblatex}
\addbibresource{references.bib}
\geometry{margin=1in}

\title{\LARGE \bfseries Establishing a Satellite Network System with Integrated LLMs for Commercial Operations in LEO}
\author{\LARGE Team: \texttt{Ordinary\_subspace} \\ Ashiqur Rahman,Ash Shafi, Md Golam Rabbani, Md. Badruddin Tasnim, Ruhul Amin Pappo}
\date{\today}

\begin{document}
\maketitle

\begin{abstract}
This document presents a commercial and technically-grounded business plan for a modular Low Earth Orbit (LEO) satellite network that embeds efficient on-board Large Language Models (LLMs) and machine learning (ML) pipelines to deliver actionable, low-latency data products \parencite{nasaopendata}. The core value proposition is to convert raw satellite sensor outputs into high-margin insights onboard, minimizing downlink costs and enabling new product lines: subscription analytics, on-demand enterprise tasking, hosted payload revenues, and an on-orbit compute marketplace. The plan includes a detailed market and unit-economics analysis, a three-phase deployment roadmap, network and software architecture, prototype specification, regulatory and debris-mitigation approaches (following NASA ODPO guidance \parencite{odpo} and DAS best practices \parencite{das}), and measurable KPIs for investor evaluation.
\end{abstract}


\section{Executive summary}
Our company, Ordinary\_subspace, will deploy a vertically-integrated LEO service offering that combines hardware, edge AI, ground/cloud services, and marketplace capabilities. Target customers include agriculture and forestry firms, insurance and reinsurance companies, infrastructure operators, humanitarian/NGO organizations, and defense/space agencies seeking near-real-time analytics with predictable costs and strong sustainability credentials \parencite{nasa_commercial}. We project a path to positive gross margin by Year 2 of commercial operations and aim for break-even on initial constellation CAPEX by Year 4 under base-case assumptions.

\section{Market opportunity and positioning}
\subsection{Total addressable market (TAM), serviceable available market (SAM), and serviceable obtainable market (SOM)}
\begin{itemize}
  \item Estimated global TAM for commercial satellite data and analytics by 2030: \num{60e9} USD (satellite communications, Earth observation, and data services combined).
  \item SAM focusing on actionable EO analytics and mission tasking: \num{12e9} USD.
  \item SOM (target first 5 years, niche markets: agriculture, insurance, infrastructure): \num{300e6} USD annual revenue target by Year 5.
\end{itemize}

\subsection{Target customer segments and willingness to pay}
\begin{tabularx}{\textwidth}{l X r}
\toprule
Segment & Use case & Indicative WTP (annual) \\
\midrule
Agriculture & Crop health monitoring, yield forecasting & \num{10e3}--\num{200e3} \\
Insurance/Reinsurance & Rapid damage assessment, claims triage & \num{50e3}--\num{1e6} \\
Infrastructure operators & Asset monitoring (pipelines, power lines) & \num{20e3}--\num{500e3} \\
NGOs / Disaster Response & Rapid situational awareness & \num{5e3}--\num{100e3} \\
Space/Defense partners & Hosted payloads, on-orbit compute & \num{100e3}--\num{5e6} \\
\bottomrule
\end{tabularx}

\section{Value proposition}
\begin{itemize}
  \item Reduce customer data costs by up to \num{70}\% through onboard filtering, prioritization, and inference that sends only actionable outputs rather than raw imagery.
  \item Enable sub-hour response times for event-driven tasks by using onboard LLM agents to autonomously detect and prioritize events.
  \item Offer a marketplace for third-parties to run validated inference pipelines on orbit, creating a new revenue channel with high margin per compute cycle.
  \item Maintain sustainability and compliance credentials using NASA ODPO guidance and formal debris assessment during design and operations \parencite{odpo}.
\end{itemize}

\section{Product and service portfolio}
\begin{enumerate}
  \item \textbf{Subscription Analytics (Core)}: Tiered monthly plans providing daily/weekly analytics, alerts, and dashboards. Pricing tiers: Basic (\$1k/month), Professional (\$7.5k/month), Enterprise (custom, typical \$50k+/month).
  \item \textbf{On-Demand Tasking}: Pay-per-task model where customers request a prioritized revisit or higher-resolution capture. Pricing: \$500--\$10,000 per task depending on latency and resolution.
  \item \textbf{Data Licensing}: Aggregated historical datasets and labeled analytics sold via enterprise licenses (\$20k--\$500k annually). We will integrate and augment open datasets (Sentinel/Landsat) for model training and product validation \parencite{copernicus,earthexplorer}.
  \item \textbf{Hosted Payloads and Virtual Payload Leasing}: Hardware hosting or virtualized instrument access on our buses for strategic partners. Hosting fees: \$250k--\$2M per hosted payload lifecycle.
  \item \textbf{On-Orbit Compute Marketplace}: Hourly or per-inference pricing for validated ML workloads executed onboard. Pricing: \$50--\$500 per compute-hour equivalent.
  \item \textbf{Professional Services}: Custom ML model development, integration and SLAs charged hourly or project-based (typically \$50k--\$300k per engagement).
\end{enumerate}

\section{Business model and unit economics}
\subsection{Key assumptions (base case)}
\begin{itemize}
  \item Smallsat unit cost (bus + payload): \$600k initial unit cost (volume discounts reduce to \$350k by scale).
  \item Launch cost per satellite (rideshare): \$200k initial, decreasing to \$100k with negotiation and bulk launches.
  \item Satellite lifetime in LEO (design): 5 years operational life; controlled deorbit after mission.
  \item Average monthly subscription ARPU: \$3,000.
  \item Average on-demand task price: \$2,500.
  \item Onboard compute operating cost (amortized): equivalent \$200/month per satellite when utilized.
  \item Customer acquisition cost (CAC): \$6,000 average (digital + events + partnerships).
  \item Gross margin on subscription services: 65\% after ground/cloud and ops.
\end{itemize}

\subsection{Unit economics and payback}
\begin{tabularx}{\textwidth}{X r r}
\toprule
Metric & Value & Notes \\
\midrule
Total cost per satellite (capex incl. launch) & \$850,000 & \$600k + \$200k + integration \\
Annualized capex per sat (5-year life) & \$170,000 & Straight-line amortization \\
Required annual subscription revenue per sat for gross breakeven & \$260,000 & assumes 65\% gross margin and \$170k amortized capex \\
Equivalent number of average ARPU customers per sat & 9 & \$260k / (\$3k*12) \\
\bottomrule
\end{tabularx}

\subsection{5-year financial summary (simplified)}
\begin{tabularx}{\textwidth}{l r r r r r}
\toprule
Year & Satellites Deployed & Revenue (\$M) & Gross Margin (\$M) & OPEX (\$M) & Net (approx) (\$M) \\
\midrule
1 & 3 & 0.6 & 0.39 & 0.5 & -0.61 \\
2 & 12 & 3.5 & 2.28 & 1.2 & 0.48 \\
3 & 30 & 10.5 & 6.83 & 2.5 & 2.33 \\
4 & 60 & 25.0 & 16.25 & 5.0 & 7.25 \\
5 & 100 & 55.0 & 35.75 & 9.0 & 18.75 \\
\bottomrule
\end{tabularx}

\section{Go-to-market strategy}
\subsection{Phased sales approach}
\begin{enumerate}
  \item Proof-of-concept sales: target 10 pilot customers across agriculture, insurance, and NGOs within first 12 months. Offer discounted pilot pricing for 6-month trials.
  \item Enterprise sales: hire dedicated account execs for insurance and infrastructure verticals to secure multi-year contracts.
  \item Channel partnerships: integrate with GIS and agritech platforms (resellers / system integrators) to expand reach and reduce CAC.
  \item Government \& institutional contracts: pursue R\&D grants, cooperative agreements, and public procurement for credibility and recurring revenue \parencite{nasa_commercial}.
\end{enumerate}

\subsection{Customer acquisition and marketing mix}
\begin{itemize}
  \item Digital: content marketing, targeted LinkedIn campaigns, webinars. Expected CAC contribution: 40\%.
  \item Events and trade shows: 30\% of CAC; enterprise leads from conferences.
  \item Partnerships \& referrals: 20\% of CAC with lower conversion time.
  \item Direct enterprise sales and tender responses: 10\% of CAC but highest ACV.
\end{itemize}

\subsection{Retention, expansion and LTV}
\begin{itemize}
  \item Expected gross retention rate: 88\% annually after initial product-market fit.
  \item Average customer lifetime: 4.5 years.
  \item Lifetime value (LTV) estimate: ARPU \$3k/month, gross margin 65\% $\rightarrow$ LTV $\approx$ \$3k*12*4.5*0.65 \approx \$105,300.
  \item LTV:CAC ratio (base case): \num{17.6} assuming CAC \$6k, indicating highly attractive unit economics at scale.
\end{itemize}

\section{Network architecture and technical design}
\subsection{Constellation design}
\begin{itemize}
  \item Initial constellation: 12 satellites in two orbital planes at \num{550} km altitude, inclinations tuned for mid-latitude coverage.
  \item Scaled constellation: 60--100 satellites covering multi-revisit, with inter-satellite links (ISLs) for prioritized relay.
  \item Typical payload: multispectral imager (2--10 m GSD) and processing module with an edge inference CPU/accelerator and an LLM micro-model for tasking and natural-language interaction.
\end{itemize}

\subsection{Onboard software stack}
\begin{enumerate}
  \item Runtime: lightweight RTOS with containerized inference runtime.
  \item Edge ML/LLM: quantized transformer family (under 100M parameters) optimized for onboard inference with pruning and distillation.
  \item Data pipeline: sensor ingestion $\rightarrow$ pre-processing $\rightarrow$ inference $\rightarrow$ prioritized packetization.
  \item Autonomy layer: mission agent that accepts natural-language commands, schedules tasks, initiates onboard analytics, and manages downlink prioritization.
\end{enumerate}

\section{Prototype specification}
\subsection{Deliverables for hackathon / investor demo}
\begin{enumerate}
  \item Web-based UI mockup showing a natural-language tasking flow, simulated satellite telemetry, and an analytics dashboard with alerts.
  \item Lightweight orbital and communications simulator (Python-based) that models revisit frequency, downlink bandwidth, and onboard compute load. Use representative parameters: downlink per pass \num{100} MB nominal raw; post-inference prioritized packet \num{0.7} MB average.
  \item Demonstration dataset: curated Sentinel/Landsat scenes and synthetic event labels to showcase on-orbit filtering (e.g., flood detection reduces transmission by \num{95}\% for non-event scenes). Prototype imagery and near-real-time visualization will use NASA Worldview and Sentinel tiles \parencite{worldview,copernicus,earthexplorer,nasaopendata}.
  \item Debris-check module: scenario-based demonstration that uses published ODPO guidelines and DAS methodology to verify disposal strategy and probability-of-collision checks in simulation \parencite{odpo,das}.
\end{enumerate}

\subsection{Prototype technical stack}
\begin{itemize}
  \item Front-end web: React + Mapbox for UI mockups and interactive maps.
  \item Simulation and server: Python (NumPy, pandas), Flask for API demo, and simple scheduler to emulate satellite passes.
  \item ML models: PyTorch or TensorFlow for training; ONNX for model quantization and inference runtime simulation.
  \item Data sources: NASA Worldview tiles and Sentinel/Landsat public imagery for realistic visuals and demonstration outputs \parencite{worldview,copernicus,earthexplorer}.
\end{itemize}

\section{Operational resilience and sustainability}
\subsection{Debris mitigation and end-of-life plan}
\begin{itemize}
  \item Controlled deorbit plan to ensure 90\% of satellites undergo reentry within 5 years of EOL \parencite{odpo}.
  \item Passivation of energy sources and stored propellant post-mission.
  \item Operational conjunction assessment with partner data feeds and automated maneuver execution for high-probability conjunctions; use of DAS/ORDEM-based analyses for licensing \parencite{das,odpo}.
  \item Regular use of ODPO models during design and DAS-style assessments for mission licensing and environmental reporting \parencite{odpo,das}.
\end{itemize}

\subsection{Reliability and redundancy}
\begin{itemize}
  \item Redundant on-board compute modules and fault detection, isolation and recovery (FDIR) logic.
  \item Ground station diversity: multi-region commercial ground station partners to ensure continuous command and data delivery.
  \item Continuous integration / continuous delivery (CI/CD) and on-orbit software update capability with staged rollouts and safety gates.
\end{itemize}

\section{Regulatory strategy and partnerships}
\subsection{Licensing and compliance}
\begin{itemize}
  \item Frequency coordination and filings with national regulators and ITU for global operations.
  \item Orbital debris and environmental compliance reports based on DAS/ORDEM outputs during licensing \parencite{das,odpo}.
  \item Export controls and data-handling policy for sensitive customers; approach includes dual-use classification review where necessary.
\end{itemize}

\subsection{Strategic partnerships}
\begin{itemize}
  \item Launch providers: secure rideshare agreements to optimize per-sat launch cost.
  \item Ground station networks and cloud partners to bake-in global downlink capacity and scalable processing.
  \item Data providers and integrators: ESA Copernicus, USGS Landsat archives for training/validation; GIS vendors for distribution \parencite{copernicus,earthexplorer}.
\end{itemize}

\section{Risk analysis and mitigation}
\subsection{Technical risks}
\begin{itemize}
  \item Onboard LLM performance and verification: mitigate via quantized model families, extensive ground testing, and conservative autonomy scope for mission-critical decisions.
  \item Hardware reliability: mitigate with iterative prototyping, redundancy, and extended environmental testing.
\end{itemize}

\subsection{Commercial risks}
\begin{itemize}
  \item Market adoption slower than expected: mitigate via pilot programs, flexible pricing, and bundling with partner offerings.
  \item Price competition: differentiate with lower TCO (total cost of ownership) achieved by onboard processing and sustainability credentials.
\end{itemize}

\section{Key performance indicators (KPIs)}
\begin{itemize}
  \item Number of paying customers (monthly active customers).
  \item ARPU and gross margin per customer.
  \item Average time-to-alert (minutes) for event detection.
  \item Downlink bandwidth saved per satellite (GB/day).
  \item Number of safe deorbits successfully executed.
  \item LTV:CAC ratio and runway months.
\end{itemize}

\section{Implementation roadmap and milestones}
\begin{tabularx}{\textwidth}{l X r}
\toprule
Phase & Activities & Timeline (months) \\
\midrule
Phase 0 & Prototype web UI, simulator, pilot integration with 3 customers & 0--6 \\
Phase 1 & Build 3 demonstration satellites, secure rideshare, run 6-month pilot & 6--18 \\
Phase 2 & Commercial constellation (12 sats), enterprise sales team & 18--36 \\
Phase 3 & Scale to 60--100 sats, deploy ISLs and marketplace & 36--60 \\
\bottomrule
\end{tabularx}

\section{Appendix: detailed financial assumptions and sensitivity}
\subsection{Sensitivity scenarios}
\begin{itemize}
  \item \textbf{Optimistic}: Launch cost reduces by 40\%, ARPU increases 25\% due to enterprise sales. Break-even in Year 3.
  \item \textbf{Base}: Figures shown in the 5-year summary above.
  \item \textbf{Conservative}: Launch costs remain high; CAC increases 50\%; break-even shifts to Year 5.
\end{itemize}

\section{References}
\nocite{*}
\printbibliography

\end{document}
